\documentclass[a4paper,14pt]{extarticle}

\usepackage[utf8x]{inputenc}
\usepackage[T1]{fontenc}
\usepackage[russian]{babel}
\usepackage{hyperref}
\usepackage{indentfirst}
\usepackage{here}
\usepackage{array}
\usepackage{graphicx}
\usepackage{grffile}
\usepackage{caption}
\usepackage{subcaption}
\usepackage{chngcntr}
\usepackage{amsmath}
\usepackage{amssymb}
\usepackage{pgfplots}
\usepackage{pgfplotstable}
\usepackage[left=2cm,right=2cm,top=2cm,bottom=2cm,bindingoffset=0cm]{geometry}
\usepackage{multicol}
\usepackage{multirow}
\usepackage{titlesec}
\usepackage{listings}
\usepackage{color}
\usepackage{longtable}
\usepackage{enumitem}
\usepackage{cmap}
\usepackage{tikz}

\usetikzlibrary{shapes,arrows}

\definecolor{green}{rgb}{0,0.6,0}
\definecolor{gray}{rgb}{0.5,0.5,0.5}
\definecolor{purple}{rgb}{0.58,0,0.82}

\lstset{
	extendedchars=\true,
	keepspaces=true,
	language={bash},
	inputpath={logs},
	backgroundcolor=\color{white},
	commentstyle=\color{green},
	keywordstyle=\color{blue},
	numberstyle=\scriptsize\color{gray},
	stringstyle=\color{purple},
	basicstyle=\small,
	breakatwhitespace=false,
	breaklines=true,
	captionpos=b,
	keepspaces=true,
	numbers=left,
	numbersep=5pt,
	showspaces=false,
	showstringspaces=false,
	showtabs=false,
	tabsize=8,
	frame=single,
}

\renewcommand{\le}{\ensuremath{\leqslant}}
\renewcommand{\leq}{\ensuremath{\leqslant}}
\renewcommand{\ge}{\ensuremath{\geqslant}}
\renewcommand{\geq}{\ensuremath{\geqslant}}
\renewcommand{\epsilon}{\ensuremath{\varepsilon}}
\renewcommand{\phi}{\ensuremath{\varphi}}
\renewcommand{\thefigure}{\arabic{figure}}
\def\code#1{\texttt{#1}}

\titleformat*{\section}{\large\bfseries} 
\titleformat*{\subsection}{\normalsize\bfseries} 
\titleformat*{\subsubsection}{\normalsize\bfseries} 
\titleformat*{\paragraph}{\normalsize\bfseries} 
\titleformat*{\subparagraph}{\normalsize\bfseries} 

\counterwithin{figure}{section}
\counterwithin{equation}{section}
\counterwithin{table}{section}
\newcommand{\sign}[1][5cm]{\makebox[#1]{\hrulefill}}
\newcommand{\equipollence}{\quad\Leftrightarrow\quad}
\newcommand{\no}[1]{\overline{#1}}
\graphicspath{{figs/}}
\captionsetup{justification=centering,margin=1cm}
\def\arraystretch{1.3}
\setlength\parindent{5ex}
\titlelabel{\thetitle.\quad}

\setitemize{itemsep=0em}
\setenumerate{itemsep=0em}

\tikzstyle{startstop} = [
	rectangle,
	align=center,
	rounded corners,
	text width=10em,
	text centered,
	draw=black
]
\tikzstyle{process} = [
	rectangle,
	align=center,
	text width=20em,
	text centered,
	draw=black
]
\tikzstyle{decision} = [
	diamond,
	aspect=4,
	align=center,
	inner sep=0pt,
	text width=10em,
	text centered,
	node distance=5em,
	draw=black
]
\tikzstyle{line} = [
	draw=black,
	thick,
	->,
	>=stealth,
	-latex'
]

\begin{document}

\begin{titlepage}
\begin{center}
	САНКТ-ПЕТЕРБУРГСКИЙ ПОЛИТЕХНИЧЕСКИЙ УНИВЕРСИТЕТ\\ ПЕТРА ВЕЛИКОГО\\[0.3cm]
	\par\noindent\rule{10cm}{0.4pt}\\[0.3cm]
	Институт компьютерных наук и технологий \\[0.3cm]
	Кафедра компьютерных систем и программных технологий\\[4cm]
	
	Отчет по лабораторной работе\\[3mm]
	Дисциплина: <<Защита информации>>\\[3mm]
	Тема: <<Рассмотрение PGP системы GnuPG>>\\[7cm]
\end{center}

\begin{flushleft}
	\hspace*{5mm} Выполнил студент гр. 43501/3  \hspace*{1.5cm}\sign[3cm]\hfill А.В. Кнорре\\
	\hspace*{9.4cm} (подпись)\\[3mm]
	\hspace*{5mm} Преподаватель \hspace*{5.0cm}\sign[3cm]\hfill А.Г. Новопашенный\\
	\hspace*{9.4cm} (подпись)\\[5mm]
	\hspace*{11.1cm} <<\sign[7mm]>> \sign[27mm] \the\year\hspace{1mm} г.
\end{flushleft}

\vfill

\begin{center}
	Санкт-Петербург\\
	\the\year
\end{center}
\end{titlepage}
\addtocounter{page}{1}

\tableofcontents
\newpage

\section{Введение}

Задачей данного реферата является изучение и анализ нововведений в федеральный закон <<Об информации, информационных технологиях и о защите информации>>, который был принят Государственной думой 8 июля 2006 года, одобрен Советом федерации 14 июля 2006 года и подписан президентом Российской Федерации 27 июля 2006 года. 


На момент 19 декабря 2018 года с даты принятия данного закона прошло более 12 лет. За этот период времени закон претерпел значительные изменения: интернет присутствует практически в каждом жилом доме, практически у всех есть мобильный телефон или смартфон с подключенными услугами мобильного интернета. Вместе с этим интернет заполнился личными данными пользователей - начиная от переписки на форуме и заканчивая личным профилем в социальной сети. Для поддержания порядка в стране было выпущено 32 изменяющих закон документа, рассмотрим их освновные аспекты и оценим их влияние на рядового пользователя сети <<Интернет>> в России.

\section{Нововведения}
\subsection{Статья 2. Основные понятия, используемые в настоящем Федеральном законе}

Статья 2 в период с 2012 года по 2015 год была дополнена определениями следующих понятий: 

\begin{enumerate}
	\item электронный документ;
	\item сайт в сети <<Интернет>>;
	\item страница сайта в сети <<Интернет>> (далее также -- интернет-страница);
	\item доменное имя;
	\item сетевой адрес;
	\item владелец сайта в сети <<Интернет>>;
	\item провайдер хостинга;
	\item единая система идентификации и аутентификации;
	\item поисковая система.
\end{enumerate} 

Рост числа сайтов, принадлежащих частным лицам, с неправомерным содержанием привёл к необходимости точечных блокировок интернет ресурсов. Для этого необходимо установить какая часть сети <<Интернет>> должна быть заблокирована в ходе постановления суда. В этом помогают: провайдер хостинга - как исполнитель блокировки; владелец сайта - как виновник присутствия неправомерного контента; доменное имя и сетевой адрес - как идентификатор ресурса в сети <<Интернет>>.

\subsection{Статья 7. Общедоступная информация}

В статье 7 в 2013 году были введены несколько новых частей. В них дано определение понятию <<информация, размещаемая в форме открытых данных>> и рассказывается, какая информация не может быть размещена в сети <<Интернет>> в форме открытых данных. В частности, если информация размещена в форме открытых данных, и она может повлечь распространение государственной тайны, то её размещение должно быть прекращено по требованию специализированного органа. Если размещаемая в форме открытых данных информация нарушает права обладателей информации или нарушает требования Федерального закона <<О персональных данных>>, то её размещение так же должно быть прекращено по требованию суда или иного уполномоченного органа.


Интернет-ресурсы реализуют разделение доступа информации по личностному признаку. Таким образом, необходимо разделять законное предоставление информации субъекту её обладания и размещение информации на странице, доступной каждому пользователю. Интересным является случай с облачными хранилищами, когда индексатор службы Google проникал в данные пользователей, ничего об этом не подозревающих. Как результат, отправив некоторый запрос в поисковую службу, можно было обнаружить данные некоторого пользователя.

\subsection{Статья 8. Право на доступ к информации}

Согласно изменениям 8 статьи в 2010 году, государственные органы и органы местного самоуправления обязаны не просто обеспечивать доступ к информации о своей деятельности, а обязаны делать это с использованием сети <<Интернет>>.


Данное изменение очень важно, так как повышает эффективность независимых расследований граждан предоставляя им доступ ко многим нормативным документам в месте, где бюрократия в форме бумажной канцелярии не может затянуть процесс получения искомых данных на недели.

\subsection{Статья 9. Ограничение доступа к информации}

В статью 9 в 2017 году была введена часть, в которой Роскомнадзору разрешается определять порядок идентификации информационных ресурсов в целях принятия мер по ограничению доступа к информационным ресурсам, требования к способам (методам) ограничения такого доступа, применяемым в соответствии с настоящим Федеральным законом, а также требования к размещаемой информации об ограничении доступа к информационным ресурсам.


К сожалению, при отсутствии у блокируемого ресурса статического сетевого адреса, государственные органы могут принять решение о блокировке целой подсети, внутри которой также находятся правомерные ресурсы. Нанесённые убытки частным и юридическим лицам никто не компенсирует.

\subsection{Статья 10. Распространение информации или предоставление информации}

В статье 10 в 2014 году часть 2 была дополнена текстом следующего содержания: Владелец сайта должен разместить на нём своё имя, местоположение и адрес электронной почты.


\subsection{Статья 10.1. Обязанности организатора распространения информации в сети <<Интернет>>}

Данная статья была целиком введена в 2014 году. В ней рассказывается, что организатор распространения информации, в частности, обязан:

\begin{enumerate}
	\item уведомить Роскомнадзор о начале осуществления деятельности;
	\item всю имеющуюся иинформацию о пользователях хранить на территории России;
	\item предоставить эту информацию уполномоченным государственным органам, осуществляющим оперативно-разыскную деятельность или обеспечение безопасности, если им она понадобится;
	\item при использовании для приёма, передачи, доставки или обработки электронных сообщений шифрования предоставлять в федеральный орган исполнительной власти в области обеспечения безопасности необходимую для расшифровки этих сообщений информацию.
\end{enumerate}


Данная статья произвела много шума, ведь рассекречивание частной переписки наносит значительный ущерб правам личности, а также является поводом для схем коммерческой направленности с косвенным участием денежных сумм и государственных органов. По этой причине многие предоставители услуг полностью переехали на зарубежные хостинги.

\subsection{Статья 10.3. Обязанности оператора поисковой системы}

Данная статья была введена в 2015 году. В ней рассказывается о том, в каких случаях операторы поисковых систем (далее поисковики) обязаны прекратить выдачу сведений о тех или иных страницах сайтов в сети <<Интернет>> (далее просто -- страницах). 


Интересным моментом является, что для прекращения выдачи поисковиками страниц не нужно решение суда, а достаточно лишь правильно сформированного заявления со стороны физического лица. Плохо обоснованная статья по той причине, что раз уж сайт доступен по прямой ссылке и индексирующая служба смогла к нему обратиться, то почему поисковая служба частной компании обязана исключить её по чьему-то запросу?


\subsection{Статья 10.5. Обязанности владельца аудиовизуального сервиса}

Статья 10.5 была введена в 2017 году. В ней для владельцев аудиовизуальных сервисов вводится запрет на распространение запрещенной информации, такой как государственная тайна. Помимо этого владалец обязан установить на вычислительные машины одну из предлагаемых Роскомнадзором программ, осуществляющих подсчёт числа пользователей сервиса.


\subsection{Статья 12.1. Особенности государственного регулирования в сфере использования российских программ для электронных вычислительных машин и баз данных}

Статья была введена в 2015 году. В ней рассказано о создании единого реестра российских программ для электронных вычислительных машин и баз данных (далее -- реестра российского программного обеспечения). Также в статье описаны основания для включения в реестр, процедура включения в реестр, процедура исключения из реестра.


Для включения в реестр программное обеспечение должно обладать следующими требованиями:

\begin{enumerate}
	\item исключительное право на программу на территории всего мира и на весь срок действия исключительного права принадлежит одному либо нескольким из следующих лиц (правообладателей):
	\begin{enumerate}
		\item[1)] Российской Федерации, субъекту Российской Федерации, муниципальному образованию;
		\item[2)] российской некоммерческой организации, решения которой иностранное лицо не имеет возможности определять;
		\item[3)] российской коммерческой организации, в которой суммарная доля Российских организаций более 50 процентов;
		\item[4)] гражданину России.
	\end{enumerate}
	\item программа уже введена в гражданский оборот;
	\item общая сумма выплат по лицензионным и иным договорам иностранным лицам менее тридцати процентов;
	\item сведения о программе не являются государственной тайной.
\end{enumerate}

\subsection{Статья 13. Информационные системы}

В этой статье основное нововведение было внесено в 2014 году. Согласно нему все технические средства информационных систем государственных организаций должны располагаться на территории России.


Данная статья имеет своей целью осложнение способов защиты сетевого трафика от расшифровки государственными службами, а также уменьшение рисков неправомерного взаимодействия с пересылаемой информацией третьими лицами, что особенно важно для государственных организаций.

\subsection{Статья 14. Государственные информационные системы}

Статья 14 изменялась и дополнялась в период с 2010 года по 2018 год. Но основные изменения произошли в 2010 и 2013 годах. Согласно ним, информация, содержащаяся в информационных системах государственных организаций, считается официальной. Также Правительство России было наделено правом определять доступ к каким государственных информационным ресурсам должен предоставляться лишь пользователям, прошедшим авторизацию в единой системе идентификации и аутентификации.

\subsection{Статья 14.1. Применение информационных технологий в целях идентификации граждан Российской Федерации}

Данная статья была введена в 2017 году. В ней рассказывает о том, какие персональные данные пользователей информационных систем должны собирать и хранить такие структуры, как: государственные органы, банки и иные организации. Также формулируется кто и как контролирует правильность хранения и обработки этих персональных данных.

Интересным является то, что данные размещаются в единой системе идентификации и аутентификации и в единой биометрической системе  и подписываются усиленной квалифицированной электронной подписью уполномоченной организации. 


Если гражданин хочет предоставить организации свои персональные данные посредством сети <<Интернет>>, то есть без личного присутствия, то для этой цели должны использоваться шифровальные (криптографические) средства, позволяющие обеспечить безопасность передачи данных. Если пользователь отказывается использовать шифровальные средства для этой цели, то организация обязана сначала ему отказать и объяснить риски, связанные с таким отказом. Далее, если пользователь подтверждает, что ознакомлен со всеми рисками и принимает их, то его идентификация может быть осуществлена без использования шифровальных средств.

\subsection{Статья 15.1. Единый реестр доменных имен, указателей страниц сайтов в сети <<Интернет>> и сетевых адресов, позволяющих идентифицировать сайты в сети <<Интернет>>, содержащие информацию, распространение которой в Российской Федерации запрещено}

В статье 15.1, которая была введена в 2012 году и претерпевала изменения вплоть до 2018 года, рассказывается о создании реестра, в котором перчислены сетевые адреса, доменные имена и указатели страниц сайтов, которые распространяют запрещённую в России информацию. В статье описаны основания для включения в реестр, процедура включения в реестр, процедура исключения из реестра, аналогично статье 12.1. 

Период времени в одни сутки после получения уведомления даётся владельцам сайта, распростроняющего запрещённую информацию. После этого ресурс помещается в реестр запрещенных сайтов. Провайдер хостинга в течение суток обязан ограничить доступ к такому сайту. Не позднее, чем в течение трёх дней с момента удаления владельцем сайта запрещённой информации и уведомлении об этом соответствующего органа, идентификатор сайта должен быть удалён из реестра.

Нововведением 2018 года является новое основание для включение в реестр -- постановление судебного пристава-исполнителя об ограничении доступа к информации, распространяемой в сети <<Интернет>>, порочащей честь, достоинство или деловую репутацию гражданина либо деловую репутацию юридического лица.

\subsection{Статья 15.3. Порядок ограничения доступа к информации, распространяемой с нарушением закона}

Статья 15.3 была введена в 2013 году, а в 2017 году претерпела сильные изменения. В данной статье уточнено какую информацию нужно считать запрещённой на территории России (призывы к массовым беспорядкам, осуществлению экстремистской деятельности, участию в массовых (публичных) мероприятиях, проводимых с нарушением установленного порядка, информационных материалов иностранной или международной неправительственной организации, деятельность которой признана нежелательной на территории России). 


Кроме того в ней описаны полномочия Роскомнадзора и порядок его действий в случае обнаружения в сети <<Интернет>> информации, распространяемой с нарушением закона.

\subsection{Статья 15.5. Порядок ограничения доступа к информации, обрабатываемой с нарушением законодательства Российской Федерации в области персональных данных}

В 2014 году в закон была введена статья 15.5, в которой рассказано о создании реестра нарушителей прав субъектов персональных данных (далее - реестр нарушителей). В статье описаны основания для включения в реестр, процедура включения в реестр, процедура исключения из реестра, аналогично статьям 12.1 и 15.1. 

\subsection{Статья 15.6-1. Порядок ограничения доступа к копиям заблокированных сайтов}

Статья 15.6-1 была введена в 2017 году. В ней описан порядок признания сайтов копиями заблокированных по причине распространения запрещенной в России информации. После признания сайта таковым, доступ к нему незамедлительно ограничивается без уведомления владельца сайта. До введения этой статьи порядок ограничения доступа к копиям заблокированных сайтов был таким же, как и порядок ограничения доступа к оригиналам заблокированных сайтов, что требовало больше времени.

\subsection{Статья 15.8. Меры, направленные на противодействие использованию на территории Российской Федерации информационно-телекоммуникационных сетей и информационных ресурсов, посредством которых обеспечивается доступ к информационным ресурсам и информационно-телекоммуникационным сетям, доступ к которым ограничен на территории Российской Федерации}

Статья 15.8 появилась в 2017 году. В данной статье запрещается владельцам сайтов в сети <<Интернет>>, владельцам информационно-телекоммуникационных сетей и информационных ресурсов предоставлять на территории России их информационно-телекоммуникационные сети и информационные ресурсы для получения доступа к сайтам, доступ к которым ограничен на территории России. 


Примером таких структур являются предоставители прокси-серверов, которые позволяют обходить блокировки на территории России. Также в этой статье описан порядок процедур по уведомлению владельцев таких информационно-телекоммуникационных сетей и информационных ресурсов о нарушении ими закона, а также сроки, в которые стороны должны выполнить предписанные законом действия.

\subsection{Статья 17. Ответственность за правонарушения в сфере информации, информационных технологий и защиты информации}

Статья 17 была дополнена частью 1.1 в 2017 году, в которой рассказывается, что лица, виновные в нарушении требований статьи 14.1 (банки, государственные и иные организации) в части обработки, включая сбор и хранение, биометрических персональных данных, несут административную, гражданскую и уголовную ответственность. 


В 2013 году была введена часть 4, согласно которой провайдер хостинга, оператор связи или владелец сайта в сети <<Интернет>> не несут ответственность перед правообладателем и перед пользователем за ограничение доступа к информации и (или) ограничение ее распространения. Логичным и обоснованным продолжением данной политики было бы предоставление владельцами сайтов инструментария Роскомнадзору по блокировке конкретных ресурсов внутри сайта, что позволило бы в случае неспособности модераторов сайта реагировать в 
течение суток избежать полной блокировки сайта путём активизации соответствующего инструментария и удаления Роскомнадзором конкретного элемента сайта, противоречащего букве закона.

\section{Заключение}


На момент 19 декабря 2018 года с даты принятия данного закона прошло более 12 лет, сфера информационных технологий претерпела значительные изменения, а распространённость доступа к сети <<Интернет>> постоянно растёт.


Нововведения в 149 федеральный закон внесли некоторую дополнительную конкретику в его статьи. Также был создан целый ряд новых обязанностей у пользователей и провайдеров сети <<Интернет>>. Модерация сетевых ресурсов усложнилась, теперь необходимо регулярное наблюдение за наполнением сайтов, а со стороны владельцев сайтом требуется еще и быстрая - в течение суток - реакция в виде удаления неправомерного содержания.


Государственные органы хотят повысить подконтрольность сетевого пространства, что логично и легко объяснимо. Однако одной из главных идей сети <<Интернет>> как развития военной технологии США является факт того, что она никому не принадлежит. Уходя от свободного доступа к сетевым ресурсам по соображениям правительства нужно быть крайне осторожным. Важно понимать, что в сети <<Интернет>> содержится множество важной информации для деятелей самых разных научных и практических направленностей, и чтобы не сгубить информационные структуры и не прекратить поток информации как образовательный процесс населения собственной страны необходима экспертная оценка принимаемых мер по цензуре в сети <<Интернет>>. К сожалению практика экспертных оценок в России плохо себя зарекомендовала, ведь в отличии от ряда других стран, у нас любой сертифицирующий, валидирующий, верифицирующий, экспертно оценивающий орган, решение которого должно повлиять на мнение государственных органов, вынужденно принадлежит государству, что делает невозможным независимую оценку по дополнительному ряду причин. 


Поэтому даже не смотря на тот факт, что в принятии решений по цензуре сети <<Интернет>> во благо народа (детская порнография, экстремизм и прочее содержание, которое ненавистно каждому гражданину страны) учавствуют люди, хорошо разбирающиеся в теме, многие последствия этих решений не учитываются. Гонка за не существование права на тайную переписку в сети противоречит самой идее сети <<Интернет>>. Однако никакая идея не интересует государство так, как практическая значимость возможности рассекретить террориста и экстремиста в режиме реального времени. И здесь мы подходим к самому главному: нужно ли сети <<Интернет>> государство, или наоборот, государству нужна сеть <<Интернет>>? Как далеко можно зайти не утратив иллюзию контроля, не надломив шпоры об отвердевшую длань?


Практическим выводом является то, что для обеспечения собственной безопасности, при нахождении на территории России, следует распологать свои сервера зарубежом во избежание конфликтов с нынешними и будущими местными порядками и законами. Также важно понимать, что практически любые действия в сетевых ресурсах, предоставляемых серверами на территории России, оставляют следы, и могут быть рассмотрены на факт правомерности согласно текущим законам Российской Федерации, что следует иметь ввиду при использовании функциональности по "личной" переписке в социальных сетях или мессенджерах, которые государство не пытается заблокировать на текущий момент.

\section{Список использованных источников}

\begin{enumerate}
	\item[[ 1]]  Федеральный закон от 27.07.2006 N 149-ФЗ (ред. от 19.07.2018) <<Об информации, информационных технологиях и о защите информации>> [Электронный ресурс]. — URL: https://fstec.ru/tekhnicheskaya-zashchita-informatsii/dokumenty/107-zakony/364-federalnyj-zakon-ot-27-iyulya-2006-g-n-149-fz (дата обращения 19.12.2018)
	\item[[ 2]]  Федеральный закон от 27.07.2006 N 149-ФЗ (ред. от 19.07.2018) <<Об информации, информационных технологиях и о защите информации>> [Электронный ресурс]. — URL: \\http://www.consultant.ru/document/cons\_doc\_LAW\_61798/ (дата обращения 19.12.2018)
\end{enumerate}

\end{document}
