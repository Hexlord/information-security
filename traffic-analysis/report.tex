\include{settings}

\begin{document}

\include{titlepage}

\tableofcontents
\listoffigures
\lstlistoflistings
\newpage

\section{Цель работы}

Научиться анализировать сетевой трафик при помощи программы \code{WireShark}.

\section{Программа работы}

Проанализировать сетевой трафик:

\begin{enumerate}
	\item Протокола ARP
	\item Протокола ICMP
	\begin{itemize}
		\item \code{ping} без фрагментации
		\item \code{ping} с фрагментацией
		\item \code{traceroute}
		\item ошибка 3/1
	\end{itemize}
	\item Протокола UDP
	\item Протокола TCP
	\begin{itemize}
		\item установление соединения
		\item закрытие соединения
		\item флаг \code{RST}
	\end{itemize}
\end{enumerate}

\section{Условия сети}

Часть работы была выполнена в сети кафедры. Её конфигурация представлена в листинге \ref{lst:net-kspt}.

\lstinputlisting[caption={Конфигурация сети кафедры},label={lst:net-kspt},keywords={ifconfig},basicstyle=\scriptsize]{ifconfig-kspt.txt}


Другая часть работы была выполнена в домашней сети, конфигурация которой представлена в листинге \ref{lst:net-home}.

\lstinputlisting[caption={Конфигурация домашней сети},label={lst:net-home},keywords={ifconfig},basicstyle=\scriptsize]{ifconfig-home.txt}

\section{Протокол ARP}

Рассмотрим принцип действия протокола ARP. Выполним широковещательный ARP-запрос в домашней сети, отправив \code{ping} пакет другому узлу сети (домашняя сеть) по адресу \code{192.168.0.100}.  На рис. \ref{fig:arp-req} представлен ARP-запрос.

\begin{figure}[H]
	\centering
	\includegraphics[width=1\textwidth]{arp-req}
	\caption{ARP-запрос}
	\label{fig:arp-req}
\end{figure}

Затем, узел сети по адресу \code{192.168.0.101} отправляет нам ARP-ответ, который представлен на рис. \ref{fig:arp-resp}

\begin{figure}[H]
	\centering
	\includegraphics[width=1\textwidth]{arp-resp}
	\caption{ARP-ответ}
	\label{fig:arp-resp}
\end{figure}

\section{Протокол ICMP}

\subsection{Утилита \code{ping}}

Утилита \code{ping} отправляет ICMP эхо-запрос, на который, в случае успеха приходит ICMP эхо-ответ. Если пакет не пришел за определённое время, то удалённый хост считается недостижимым.

\paragraph{Ping без фрагментации}

Т.к. для интерфейса wlp3s0 размер MTU равен 1500 байтов (без заголовка -- 1480), то для отправки пакетов утилитой \code{ping} без фрагментации ничего не нужно делать, т.к. размер пакета по умолчанию не превышает MTU.

Отправим с помощью утилиты \code{ping} ICMP эхо-запрос сайту lib.ru (листинг \ref{lst:ping}). Он представлен на рис. \ref{fig:ping-req}.

\lstinputlisting[caption={Ping без фрагментации},label={lst:ping},keywords={ping}]{ping.txt}

\begin{figure}[H]
	\centering
	\includegraphics[width=1\textwidth]{ping-req}
	\caption{Ping ICMP эхо-запрос (без фрагментации)}
	\label{fig:ping-req}
\end{figure}

Затем нам приходит от сайта lib.ru ICMP эхо-ответ (рис. \ref{fig:ping-resp}).

\begin{figure}[H]
	\centering
	\includegraphics[width=1\textwidth]{ping-resp}
	\caption{Ping ICMP эхо-ответ (без фрагментации)}
	\label{fig:ping-resp}
\end{figure}

\newpage

\paragraph{Ping с фрагментацией}

Для достижения фрагментации пакета, зададим его размер равным 4000 байтов (листинг \ref{lst:ping-frag}).

\lstinputlisting[caption={Ping с фрагментацией},label={lst:ping-frag},keywords={ping}]{ping-frag.txt}

На рис. \ref{fig:ping-list} представлен порядок отправки и получения IP пакетов.

\begin{figure}[H]
	\centering
	\includegraphics[width=0.9\textwidth]{ping-fraq-list}
	\caption{Порядок отправки и получения IP пакетов}
	\label{fig:ping-list}
\end{figure}

Как в случае ICMP эхо-запроса, так и в случае ICMP эхо ответа, ICMP заголовок находится в третьем, т.е. последнем, IP фрагменте.

Отправленные пакеты:

\begin{figure}[H]
	\centering
	\includegraphics[width=1\textwidth]{ping-req-frag1}
	\caption{Ping ICMP эхо-запрос (с фрагментацией) фрагмент 1}
	\label{fig:ping-req-frag}
\end{figure}

\begin{figure}[H]
	\centering
	\includegraphics[width=1\textwidth]{ping-req-frag2}
	\caption{Ping ICMP эхо-запрос (с фрагментацией) фрагмент 2}
	\label{fig:ping-req-frag}
\end{figure}

\begin{figure}[H]
	\centering
	\includegraphics[width=1\textwidth]{ping-req-frag3}
	\caption{Ping ICMP эхо-запрос (с фрагментацией)}
	\label{fig:ping-req-frag}
\end{figure}

Приходящий от сайта lib.ru ICMP эхо-ответ:

\begin{figure}[H]
	\centering
	\includegraphics[width=1\textwidth]{ping-resp-frag1}
	\caption{Ping ICMP эхо-ответ (с фрагментацией) фрагмент 1}
	\label{fig:ping-resp-frag}
\end{figure}

\begin{figure}[H]
	\centering
	\includegraphics[width=1\textwidth]{ping-resp-frag2}
	\caption{Ping ICMP эхо-ответ (с фрагментацией) фрагмент 2}
	\label{fig:ping-resp-frag}
\end{figure}

\begin{figure}[H]
	\centering
	\includegraphics[width=1\textwidth]{ping-resp-frag3}
	\caption{Ping ICMP эхо-ответ (с фрагментацией)}
	\label{fig:ping-resp-frag}
\end{figure}

\subsection{Утилита \code{traceroute}}\label{sec:traceroute}

Стоит отметить, что реализация трассировки под операционной системой Windows в утилите \code{tracert} базируется на протоколе ICMP -- она отправляет ICMP эхо-запросы. Для unix-подобных операционных систем существует утилита \code{traceroute}, которая отправляет UDP фрагменты и анализирует сообщения о доступности и достижимости порта. Хост генерирует UDP фрагмент, инкапсулирует его в IP пакет и выставляет время жизни TTL равное 1. Первый маршрутизатор на пути ответит на данный пакет ICMP сообщением об окончании времени жизни пакета. Тогда утилита сохранят адрес источника ICMP пакета как адрес первого хопа. Следующий пакет имеет TTL равное 2, и так далее. Также утилита указывает в UDP заголовке destination port равный 33434 и инкрементирует его на единицу с каждой попыткой. Когда пакет дошёл до сервера назначения и сервер распаковывает его, он понимает что данный порт у него закрыт, и отправляет ICMP сообщение (ICMP Type 3 <<Destination Unreachable>> Code 3 <<Port Unreachable>>). Получив данное сообщение утилита понимает, что трассировка закончена.

Выполним трассировку маршрута до сайта lib.ru (листинг \ref{lst:trace}).

\lstinputlisting[caption={Трассировка маршрута до lib.ru},label={lst:trace},keywords={traceroute}]{traceroute.txt}

На рис. \ref{fig:t1} --- \ref{fig:t8} изображены ICMP ответы четырёх первых маршрутизаторов, трёх последних маршрутизаторов, и удалённого хоста.

Последний узел отправляет 

\begin{figure}[H]
	\centering
	\includegraphics[width=1\textwidth]{t1}
	\caption{ICMP ответ маршрутизатора домашней сети}
	\label{fig:t1}
\end{figure}

\begin{figure}[H]
	\centering
	\includegraphics[width=1\textwidth]{t2}
	\caption{ICMP ответ маршрутизатора 2}
	\label{fig:t2}
\end{figure}

\begin{figure}[H]
	\centering
	\includegraphics[width=1\textwidth]{t3}
	\caption{ICMP ответ маршрутизатора 3}
	\label{fig:t3}
\end{figure}

\begin{figure}[H]
	\centering
	\includegraphics[width=1\textwidth]{t4}
	\caption{ICMP ответ маршрутизатора 4}
	\label{fig:t4}
\end{figure}

\begin{figure}[H]
	\centering
	\includegraphics[width=1\textwidth]{t5}
	\caption{ICMP ответ маршрутизатора N-3}
	\label{fig:t5}
\end{figure}

\begin{figure}[H]
	\centering
	\includegraphics[width=1\textwidth]{t6}
	\caption{ICMP ответ маршрутизатора N-2}
	\label{fig:t6}
\end{figure}

\begin{figure}[H]
	\centering
	\includegraphics[width=1\textwidth]{t7}
	\caption{ICMP ответ маршрутизатора N-1}
	\label{fig:t7}
\end{figure}

\begin{figure}[H]
	\centering
	\includegraphics[width=1\textwidth]{t8}
	\caption{ICMP ответ маршрутизатора удалённого хоста}
	\label{fig:t8}
\end{figure}

\subsection{Ошибка 3/1}

Для получения ICMP ошибки 3/1 (Хост недостижим), выполним \code{ping} хоста \code{10.1.17.3} сети кафедры. ICMP-ответ с ошибкой предствален на рис. \ref{fig:icmp31}.

\begin{figure}[H]
	\centering
	\includegraphics[width=1\textwidth]{icmp31}
	\caption{ICMP ответ 3/1}
	\label{fig:icmp31}
\end{figure}

\newpage

\section{Протокол UDP}

Развернём на хосте \code{192.168.0.101} UDP сервер на порту 27015. Затем на нашем хосте \code{192.168.0.100} запустим клиентское приложение и выполним сервер запрос. Он изображён на рис. \ref{fig:udp-req}.

\begin{figure}[H]
	\centering
	\includegraphics[width=1\textwidth]{udp-req}
	\caption{Запрос на сервер по UDP}
	\label{fig:udp-req}
\end{figure}

Сервер реализован таким образом, что на правильно сформированный запрос он отправляет ответ. Ответ представлен на рис. \ref{fig:udp-resp}.

\begin{figure}[H]
	\centering
	\includegraphics[width=1\textwidth]{udp-resp}
	\caption{Ответ UDP сервера}
	\label{fig:udp-resp}
\end{figure}

\section{Протокол TCP}

Развернём на хосте \code{192.168.0.101} TCP сервер на порту 27015. 

\subsection{Установление соединения}

На нашем хосте \code{192.168.0.100} запустим клиентское приложение. Оно сразу при запуске пытается установить соединение с сервером. Как видно на рис. \ref{fig:tcp-syn} -- \ref{fig:tcp-ack} соединение происходит за 3 шага:

\begin{enumerate}
	\item Клиент отправляет \code{SYN}
	\item Сервер отправляет \code{SYN} и \code{ACK}
	\item Клиент отправляет \code{ACK}
\end{enumerate}

\begin{figure}[H]
	\centering
	\includegraphics[width=1\textwidth]{tcp-syn}
	\caption{Отправка клиентом \code{SYN}}
	\label{fig:tcp-syn}
\end{figure}

\begin{figure}[H]
	\centering
	\includegraphics[width=1\textwidth]{tcp-synack}
	\caption{Отправка сервером \code{SYN} и \code{ACK}}
	\label{fig:tcp-synack}
\end{figure}

\begin{figure}[H]
	\centering
	\includegraphics[width=1\textwidth]{tcp-ack}
	\caption{Отправка клиентом \code{ACK}}
	\label{fig:tcp-ack}
\end{figure}

\subsection{Закрытие соединения}

На хосте \code{192.168.0.101}, на котором развёрнут сервер инициируем разрыв соединения, отправив клиенту \code{FIN} и \code{ACK}. После этого клиент отправляет серверу \code{ACK}, и соединение считается закрытым.

На рис. \ref{fig:tcp-finack} изображён TCP-сегмент с \code{FIN} и \code{ACK}.

\begin{figure}[H]
	\centering
	\includegraphics[width=1\textwidth]{tcp-finack}
	\caption{Отправка сервером \code{FIN} и \code{ACK}}
	\label{fig:tcp-finack}
\end{figure}

На рис. \ref{fig:tcp-finack-ack} изображён TCP-сегмент с \code{ACK}.

\begin{figure}[H]
	\centering
	\includegraphics[width=1\textwidth]{tcp-finack-ack}
	\caption{Отправка клиентом \code{ACK}}
	\label{fig:tcp-finack-ack}
\end{figure}

\subsection{Флаг \code{RST}}

Завершим работу TCP сервера, развёрнутого на хосте \code{192.168.0.101}. После этого запустим на нашем хосте клиентское приложение, которое сразу попытается установить соединение. Но этого сделать не удаётся -- на \code{SYN} приходит \code{RST} и \code{ACK}. TCP-сегмент с \code{RST} и \code{ACK} представлен на рис. \ref{fig:tcp-rstack}.

\begin{figure}[H]
	\centering
	\includegraphics[width=0.9\textwidth]{tcp-rstack}
	\caption{TCP-сегмент с \code{RST} и \code{ACK}}
	\label{fig:tcp-rstack}
\end{figure}

\section{Выводы}

Удалось поработать с утилитой \code{WireShark}, которая позволила рассмотреть трафик таких протоколов, как: UDP, TCP, ARP, ICMP. Применялись утилиты \code{ping} и \code{traceroute}, которые основываются на протоколе ICMP. Они позволяют определить достижимость определенного удалённого хоста, а также трассировать маршрут до него.

Утилита \code{WireShark} показала себя как практичный и удобный инструмент для анализа исходящего и входящего трафика. С её помощью также удаётся наглядно пронаблюдать, что при отсутствии шифрования пересылаемые данные очень легко прочитать, что видно, например, в TCP и UDP клиент-серверном взаимодействии использующем строковый протокол.

\end{document}
